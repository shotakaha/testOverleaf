\RequirePackage[2020/02/02]{latexrelease}

\documentclass{jlreq}

%\usepackage{jlreq-trimmarks}
\usepackage{graphicx}
\usepackage{amsmath, amssymb}
\usepackage{physics}


%% タイトル情報を入力する
\title{TestOverleaf}
\author{shotakah}
\date{\today}

\begin{document}

%% タイトルページを出力
\maketitle

\section{プロジェクトの設定}

左上のメニューを開きコンパイラーを Lua\LaTeX にし、
ドキュメントクラスは jlreq にした。
メニューでは、エディタの設定ももう少しできるみたい。

ドキュメントを変更し保存すると、自動コンパイルが走るみたい。

\section{GitHub連携}

Overleafのアカウント設定からGitHub連携する。
OverleafからGitHubへのアクセスを認証する。

\section{GitHubと同期}

メニューの Sync から GitHub を選択する。
GitHub連携の設定が終わっていれば、自分のGitHubリポジトリを作成するダイアログがでてくるので、適切な内容を入力する。
プライベートリポジトリを作ることもできる。
(これはパブリックにしている)

\end{document}
